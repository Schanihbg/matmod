\documentclass{beamer}
\usetheme{Copenhagen}
\usepackage[utf8]{inputenc}
\usepackage[swedish]{babel}

\author{Henrik Fredriksson}
\institute{Blekinge Tekniska Högskola}
\title{MA1477 Matematisk modellering \\ Informationsmöte}
\begin{document}


\maketitle
\begin{frame}
  \tableofcontents
\end{frame}

\section{Kurslitteratur \& Kursplan}

\begin{frame}{Kurslitteratur}
  \begin{itemize}
  \item Matematik 5000 3c Basåret, Alfredsson m.fl., Natur, ISBN: 978-91-27-43010-5
  \item Statistisk dataanalys, Svante Körner och Lars Wahlgren, ISBN:978-91-44-10870-4
  \end{itemize}
\end{frame}


\begin{frame}{Innehåll}
  \begin{itemize}

\item Algebra och polynom
\item Rationella uttryck
\item Funktioner
\item Förändringshstigheter och derivator
\item Sannolikhetslära
\item Statistisk slutledning
\item Linjär regression
  \end{itemize}
\end{frame}


\begin{frame}{Kunskap och förståelse}
  Axplock från kursplanen
  \begin{itemize}
  \item Man ska kunna utföra enklare matematiska omskrivningar
av uttryck, som förekommer i formler och
ekvationer.
\begin{enumerate}
\item kvadreringsreglerna, konjugatregeln, faktorisering
\item räta linjens ekvation (linjära modeller)
\item derivata
\item funktioner och funktionsgrafer
\end{enumerate}
\item Man ska visa förståelse för grundläggande begrepp inom sannolikhetsteori och statistik.
  \begin{enumerate}
  \item utfallsrum
  \item läges- och spridningsmått
  \item diskreta fördelningar
  \item Pearsons korrelationkoefficient
  \item statistiska analyser
  \end{enumerate}
\end{itemize}
\end{frame}





\section{Föreläsningar}
\begin{frame}{Föreläsningar}
  \begin{itemize}
  \item Länk till föreläsningar finns i läsanvisningarna på \texttt{https://github.com/dbwebb-se/matmod/}
  \item Om önskemål finns så kan ytterliggare material spelas in (t.ex
    lösningsförslag).
  \end{itemize}
\end{frame}

\section{Frågestunder/Räkneövningar}
\begin{frame}{Frågestunder}
  \begin{itemize}
  \item Frågestunder/räkneövningar kommer att ske både på campus och
    online.
    \begin{itemize}
    \item Campus: tisdagar 10-12
    \item Hangouts: Flexibel, förslag är tisdagar 13-15
    \item Gemensamt: torsdagar 15-17 
    \end{itemize}
  \item Posta i diskussionsforumet på
    \texttt{https://dbwebb.se/forum/viewforum.php?f=72} 
    eller Gitter om ni vill att jag förbereder lösningsförslag och dylikt till uppgifter.
  \end{itemize}
\end{frame}

\section{Inlämningsuppgifter}
\begin{frame}{Inlämningsuppgifter}
  \begin{itemize}
  \item I kursen ingår ett examinationsmoment på 3 hp som innefattar
    3 inlämningsuppgifter i Python
  \item Peer-assessment: Bedömning av studiekamraters inlämning
  \item Exempel på inlämningsuppgift: Skapa en webbsida som hämtar
    data från SMHI för att förutse morgondagens väder.
  \item Avslutande självreflektion
  \end{itemize}
\end{frame}
\section{Tentamen}
\begin{frame}{Tentamen}
  \begin{itemize}
  \item Tentamensdatum är 11 januari. Anmälan öppnar 12 december och stänger 28 decemeber
  \item Glöm inte anmäla tentamen på annan ort ifall ni ej har
    möjlighet att tentera i Karlskrona.
  \end{itemize}
  Obs! Man måste ha godkänt på både projekt och tentamen för att bli godkänd på kursen.
\end{frame}

\section{Treveckorsupprop}
\begin{frame}
  \begin{itemize}
  \item Obligatoriskt treveckorsupprop kommer att ske vecka 46-47.
  \item Mindre uppgift
  \end{itemize}

\end{frame}

\section{Övriga frågor}
\begin{frame}{Övriga frågor}
  \centering Frågor?
\end{frame}

\begin{frame}
  \centering Tack för er uppmärksamhet
\end{frame}

\end{document}

%%% Local Variables:
%%% mode: latex
%%% TeX-master: t
%%% End:
