\documentclass[a4paper]{article}
\usepackage[utf8]{inputenc}
\usepackage[T1]{fontenc}
\usepackage[swedish]{babel}
\usepackage{amsmath, amsthm}
\usepackage{mathtools}
\usepackage{amssymb}
\usepackage{bm}
\usepackage{enumerate}
\usepackage[x11names]{xcolor}
\usepackage[hyperindex,breaklinks,bookmarks=false]{hyperref}
\usepackage{systeme}
\usepackage{rotating}
\usepackage{polynom}

\hypersetup{
	colorlinks, %
	urlcolor=blue, %
	linkcolor=blue,
	citecolor=blue
}


\title{\bf MA1477 Matematisk modellering \\[10pt]  Veckotest v.
  46} 
\date{}

\begin{document}
\maketitle

\paragraph{1.} Bestäm ekvationen för den linje $y$ som går igenom
punkterna $(-3,1)$ och $(3,5)$
\paragraph{2.} Ange ekvationen för linjen $y$ som går igenom punkten
$(1,1)$ och är
\begin{enumerate}[(a)]
\item parallell med linjen $y = 4x +10$
\item vinkelrät mot linjen $y = -3x +1$
\end{enumerate}
\paragraph{3.} Låt $f(x) = 3x^2-18x-21$.
\begin{enumerate}[(a)]
\item Ange koordinaterna där grafen till $f(x)$ skär koordinataxlarna.
\item Bestäm symmetrilinje och vertex för $f(x)$.
\end{enumerate}
\paragraph{4.} Antag att funktionen $f(x) = x^2 + px +q$ har
nollställena $\alpha$ och $\beta$. Bestäm ett samband mellan $p,q$ och
$\alpha, \beta$.

\paragraph{5.} Lös ekvationen $5\cdot 10^x = 125$


\end{document}

%%% Local Variables:
%%% mode: latex
%%% TeX-master: t
%%% End:
