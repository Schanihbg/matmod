\documentclass{beamer}
\usetheme{Copenhagen}
\usepackage[utf8]{inputenc}
\usepackage[swedish]{babel}
\usepackage{minted}
\newtheorem{exempel}{Exempel}
\newtheorem{tre}{Treveckorsuppgift}


\author{Henrik Fredriksson}
\institute{Blekinge Tekniska Högskola}
\title{MA1477 Matematisk modellering \\ Summor och summanotation}
\begin{document}


\maketitle

\begin{frame}
\begin{exempel}
  Betrakta summan av de tio första naturliga heltalen, dvs summan
\[
S = 1+2+3 + 4 + 5 + 6 + 7 + 8 + 9 + 10
\]
\end{exempel}
\pause
Vi ska titta på hur vi kan använda summatecknet $\sum$ (sigma) för att
beskriva denna summa.

\pause
Summan $S = 1+2+3+4+5+6+7+8+9+10$ kan vi skriva som
\[
\sum_{i = 1}^{10} i
\]
\pause
Variablen $i$ kallas för summationsindex, $1$ är nedre gräns för
summationen, $10$ är övre gräns för summationen.
\end{frame}


\begin{frame}

Mer generellt kan vi skriva summor på formen
\[
\sum_{i = m}^n a_i = a_m + a_{m+1} + a_{m+2} + \ldots + a_n
\]

\begin{itemize}
\item $\sum$ summasymbol
\item $i$ summationsindex
\item $m$ nedre gräns
\item $n$ övre gräns
\item $a_i$ summand
\end{itemize}
\end{frame}


\begin{frame}[fragile]
\begin{minted}{python}
  def my_sum(l):
      S = 0
  for i in l:
      S = S + i
  return S
\end{minted}


  I Python så finns den inbyggda metoden \mint{python}|sum|
  som kan användas enligt
\begin{minted}{python}
  sum([1,2,3,4,5,6,7,8,9,10])
\end{minted}
eller
\begin{minted}{python}
  sum(range(11))
\end{minted}
\end{frame}

\begin{frame}{Ytterliggare exempel}
  \begin{exempel}
    Summan av kvadraterna av alla heltal mellan $1$ och $20$
    \[
      \sum_{i = 1}^{20} i^2 = 1^2 + 2^2+3^3 + \ldots + 20^2
    \]
    \pause
    Summan av alla udda heltal mellan $5$ och $17$
    \[
      \sum_{i = 2}^{8} (2i +1) = 5+ 7 + 9 +\ldots + 17
    \] 
    \pause
    Det aritmetiska medelvärdet av de $n$ talen $x_1, x_2 \ldots x_n$
    \[
      \dfrac{\sum_{i=1}^n x_i}{n} = \dfrac{x_1 + x_2 + \ldots + x_n}{n}
    \]

  \end{exempel}
\end{frame}

\begin{frame}{Treveckorsuppgift}

  \begin{tre}
    Beräkna summan
    \[
      \sum_{k = 0}^{10} (-1)^{2k+1}
    \]
  \end{tre}
  Fyll i svaret på itslearning under treveckorsupprop.
\end{frame}

\end{document}

%%% Local Variables:
%%% mode: latex
%%% TeX-master: t
%%% End:
