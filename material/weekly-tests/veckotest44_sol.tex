\documentclass[a4paper]{article}
\usepackage[utf8]{inputenc}
\usepackage[T1]{fontenc}
\usepackage[swedish]{babel}
\usepackage{amsmath, amsthm}
\usepackage{mathtools}
\usepackage{amssymb}
\usepackage{bm}
\usepackage{enumerate}
\usepackage[x11names]{xcolor}
\usepackage[hyperindex,breaklinks,bookmarks=false]{hyperref}
\usepackage{systeme}
\usepackage{rotating}
\usepackage{polynom}

\hypersetup{
	colorlinks, %
	urlcolor=blue, %
	linkcolor=blue,
	citecolor=blue
}
\newtheorem*{sol}{\normalfont\textit{Lösningsförslag}}

\title{\bf MA1477 Matematisk modellering \\[10pt]  Veckotest v.
  44 Lösningsförslag} 

\begin{document}
\maketitle

\paragraph{1.} Förenkla så långt som möjligt
\begin{enumerate}[(a)]
\item $9(x-1)^2 - (3x-1)^2 - 3(1-2x)$
  \begin{sol}
    \begin{align*}
      9(x-1)^2 - (3x-1)^2 - 3(1-2x) 
      &= 9(x^2- 2x +1) - (9x^2 - 6x +1) - (3-6x)\\[1em]
      &= 9x^2 - 18x + 9 - 9x^2 +6x -1 -3 + 6x \\[1em]
      &= -6x +5
    \end{align*}
  \end{sol}
\item $\dfrac{2x}{3} + \dfrac{2(1-x)}{6} + \dfrac{x-1}{2}$
  \begin{sol}
    \begin{align*}
      \dfrac{2x}{3} + \dfrac{2(1-x)}{6} + \dfrac{x-1}{2} 
      &= \frac{2x}{3}\cdot \frac{2}{2} + \frac{2(1-x)}{6} +
        \frac{x-1}{2} \cdot \frac{3}{3} \\[1em]
      &= \frac{4x + 2(1-x)+ 3(x-1)}{6} \\[1em]
      &= \frac{4x + 2-2x +3x-3}{6} \\[1em]
      &= \frac{5x-1}{6} \\[1em]
    \end{align*}
  \end{sol}
\item $\dfrac{27^2\cdot 9^{-1}\cdot 6^2}{3^8\cdot 3^{-2}}$
  \begin{sol}
    \begin{align*}
      \frac{27^2\cdot 9^{-1}\cdot 6^2}{3^8\cdot 3^{-2}}
      &= \frac{(3^3)^2\cdot (3^2)^{-1}\cdot (2\cdot 3)^2}{3^8\cdot
        3^{-2}} \\[1em]
      &= \frac{3^6\cdot 3^{-2}\cdot 2^2 \cdot 3^2}{3^8\cdot 3^{-2}}
      \\[1em]
      &= 2^2\cdot 3^6\cdot 3^{-2}\cdot 3^2 \cdot 3^{-8} \cdot 3^2 \\[1em]
      &= 2^2\cdot 3^{6-2+2-8+2} \\[1em]
      &= 4\cdot 3^0 \\[1em]
      & = 4
    \end{align*}
  \end{sol}
\item $\dfrac{\sqrt{50} - \sqrt{98}}{\sqrt{8}}$
  \begin{sol}
    \begin{align*}
      \frac{\sqrt{50} - \sqrt{98}}{\sqrt{8}} 
      &= \frac{\sqrt{2\cdot 25} - \sqrt{2\cdot 49}}{\sqrt{2\cdot 4}} \\[1em]
      &= \frac{\sqrt{2}\cdot \sqrt{25} - \sqrt{2}\cdot \sqrt{49}}{\sqrt{2}\cdot \sqrt{4}} \\[1em]
      &= \frac{\sqrt{2}\cdot 5 - \sqrt{2}\cdot 7}{\sqrt{2}\cdot 2} \\[1em]
      &= \frac{\sqrt{2}(5-7)}{\sqrt{2}\cdot 2} \\[1em]
      &= \frac{5-7}{2} \\[1em]
      &= -1
    \end{align*}
  \end{sol}
\end{enumerate}
\paragraph{2.} Lös ekvationen
\begin{enumerate}[(a)]
\item $4x^2 - 9 = 0$
  \begin{sol}
    \begin{align*}
      4x^2- 9 &= 0 \\[1em]
      (2x)^2- 3^2 &= 0 \\[1em]
      (2x+3)(2x-3) &= 0 \\[1em]
    \end{align*}
    Med hjälp av nollproduktsmetoden får vi att lösningarna till
    ekvationen är $x= -\frac{3}{2}$ och $x = \frac{3}{2}$
  \end{sol}
\item $x^2 + 3x -2 =0  $
  \begin{sol}
    Med $pq$-formeln får vi att
    \begin{align*}
      x &= - \frac{3}{2} \pm \sqrt{\left( \frac{3}{2} \right)^2 + 2} \\[1em]
      x &= - \frac{3}{2} \pm \sqrt{\frac{9}{4} + \frac{8}{4}} \\[1em]
      x &= - \frac{3}{2} \pm \sqrt{\frac{17}{4}} \\[1em]
      x &= - \frac{3}{2} \pm \frac{\sqrt{17}}{2} \\[1em]
    \end{align*}
    Lösningarna till ekvationen är $x = -\dfrac{3}{2} +
    \dfrac{\sqrt{17}}{2}$ och $- \dfrac{3}{2} - \dfrac{\sqrt{17}}{2}$ 
  \end{sol}
\item $(x^2-4)(x-1)(x^2 - 5x + 6) = 0$
  \begin{sol}
    \begin{align*}
      (x^2-4)(x-1)(x^2 - 5x + 6) &= 0 \\[1em]
      (x+2)(x-2)(x-1)(x^2 - 5x + 6) &= 0 \\[1em]
    \end{align*}
    Med nollproduktsmetoden ser vi att $3$ av lösningarna är $x= -2$,
    $x = 2$ och $x = 1$. Med $pq$-formeln så får vi att lösningarna
    till $x^2-5x+6 = 0$ är $x = 3$ och $x = 2$.

    Samtliga lösningar till ekvationen är $x = -2$, $x=2$ (dubbelrot), $x = 1$,
    och $x = 3$.
  \end{sol}
\end{enumerate}
\paragraph{3.} Bestäm en funktion $f(x)$ på formen $f(x) = ax^2 + bx +c$
som uppfyller $f(0) = 3, f(1) = 2$ och $f(-2) = 3$.
\begin{sol}
  Eftersom $f(0) = 3$ så får vi att
  \[
    f(0) = a\cdot0^2 +b\cdot 0 +c = 3
  \]
  vilket ger att $c = 3$. Vidare har vi att
  \begin{equation}
    \label{eq1}
    f(1) = a\cdot 1^2 + b\cdot 1 +3 = a +b +3 = 2 
  \end{equation}
  och
\begin{equation}
    \label{eq2}
    f(1) = a\cdot (-2)^2 + b\cdot (-2) +3 = 4a -2b +3 = 3.
  \end{equation}
  Från \eqref{eq1} får vi att $a = -1 -b$. Insättning av detta värde
  av $a$ i \eqref{eq2} ger
  \begin{align*}
    4a - 2b + 3 &= 3 \\[1em]
    4(-1-b) -2b &= 0 \\[1em]
    -4-4b -2b &= 0 \\[1em]
    -6b &= 4 \\[1em]
    b &= -\frac{4}{6} = -\frac{3}{2} \\[1em]
  \end{align*}
  Eftersom $a = -1 -b$ så får vi att
  \[
    a= -1 -b = -1 - \left(-\frac{3}{2} \right) = -\frac{2}{2} +
    \frac{3}{2} = \frac{1}{3}
  \]
  Den sökta funktionen är $f(x) = -\dfrac{1}{3}x^2 - \dfrac{2}{3}x + 3$
\end{sol}
\paragraph{4.} Lös ekvationen
\begin{enumerate}[(a)]
\item $x^2(x+3) - 1(x+3) = 0$
  \begin{sol}
    Bryter vi ut $(x+3)$ ur båda termerna i vänsterledet så får vi
    \begin{align*}
      x^2(x+3) - 1(x+3) &= 0\\[1em]
      (x+3)(x^2 -1) &= 0\\[1em]
    \end{align*}
    Eftersom $x^2-1 = (x+1)()x-1$ enligt konjugatregeln så får vi att
    \begin{align*}
      (x+3)(x^2 -1) &= 0\\[1em]
      (x+3)(x+1)(x-1) &= 0\\[1em]
    \end{align*}    
    Lösningarna till ekvationen är $x = -3$,$x = -1$ och $x = 1$.
  \end{sol}
\item $(x^2+2x+1)(x^2-1) + (x^2+2x +1)(2x^2 -10) = 0$
  \begin{sol}
    Bryter vi ut $x^2+2x-1$ ur båda termerna i vänsterledet så får vi
    att
    \begin{align*}
      (x^2+2x+1)(x^2-1) + (x^2+2x +1)(2x^2 -10) &= 0 \\[1em]
      (x^2+2x+1)(x^2-1) + 2x^2 -10) &= 0 \\[1em]
      (x+1)^2(3x^2 -11) = 0
    \end{align*}
    Eftersom lösningarna till $3x^2 -11 = 0$ är $x = \pm
    \sqrt{\dfrac{11}{3}}$ så är samtliga lösningar till ekvation
    $x = -1$ (dubbelrot), $x = \pm
    \sqrt{\dfrac{11}{3}}$
  \end{sol}
\end{enumerate}
\paragraph{5.} Faktorisera polynomen
\begin{enumerate}[(a)] 
\item $p(x) = 4x^2 -32x +60$
  \begin{sol}
    \begin{align*}
      p(x) &= 4x^2 -32x + 60 \\[1em]
      &= 4(x^2-8x + 15) 
    \end{align*}
    Med $pq$-formeln finner vi att nollställena till $x^2-8x+15$ är
    $3$ och $5$. Faktoriseringen är alltså
    \[
      p(x) = 4(x-3)(x-5)
    \]
  \end{sol}
\item $p(x) = 20x^2 -245$
  \begin{sol}
    \begin{align*}
      p(x) &= 20x^2 - 245 \\[1em]
      &= 5(4x^2 - 49) \\[1em]
      &= 5(2x)^2 - 7^2) \\[1em]
      &= 5(2x +7)(2x-7) \\[1em]
    \end{align*}
  \end{sol}
\end{enumerate}


\end{document}

%%% Local Variables:
%%% mode: latex
%%% TeX-master: t
%%% End:
