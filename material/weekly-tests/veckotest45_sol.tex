\documentclass[a4paper]{article}
\usepackage[utf8]{inputenc}
\usepackage[T1]{fontenc}
\usepackage[swedish]{babel}
\usepackage{amsmath, amsthm}
\usepackage{mathtools}
\usepackage{amssymb}
\usepackage{bm}
\usepackage{enumerate}
\usepackage[x11names]{xcolor}
\usepackage[hyperindex,breaklinks,bookmarks=false]{hyperref}
\usepackage{systeme}
\usepackage{rotating}
\usepackage{polynom}

\hypersetup{
	colorlinks, %
	urlcolor=blue, %
	linkcolor=blue,
	citecolor=blue
}

\newtheorem*{sol}{\normalfont\textit{Lösningsförslag}}

\title{\bf MA1477 Matematisk modellering \\[10pt]  Veckotest v.
  45 Lösningsförslag}
\date{}

\begin{document}
\maketitle

\paragraph{1.} För vilka värden på $x$ är följande rationella uttryck
ej definierat
\[
\dfrac{1}{x} + \dfrac{1}{x-1} + \dfrac{1}{x-2} + \dfrac{1}{x-3} + \dfrac{1}{x-4}
\]
\begin{sol}
  Det rationella uttrycket är ej definierat för de $x$-värden som gör
  att det blir $0$ i någon av nämnarna. Uttrycket är alltså inte
  definierat för $x = 0, x = 1, x= 2, x= 3$ och $x = 4$.
\end{sol}

\paragraph{2.} Förenkla följande rationella uttryck
\begin{enumerate}[(a)]
\item $\dfrac{2x^2}{4x^2 + 16x +16}$
  \begin{sol}
    \begin{align*}
      \dfrac{2x^2}{4x^2 + 16x +16}
      &= \frac{2x^2}{(2x)^2 + 2\cdot 2x\cdot 4 + 4^2} \\[1em]
      &= \frac{2x^2}{(2x+4)^2} \\[1em]
      &= \frac{2x^2}{4(x+2)^2} \\[1em]
      &= \frac{x^2}{2(x+2)^2} \\[1em]
    \end{align*}
  \end{sol}
\item $\dfrac{x^2-10x+25}{2x^2-50}$
  \begin{sol}
    \begin{align*}
      \dfrac{x^2-10x+25}{2x^2-50}
      &= \frac{x^2 - 2\cdot 5x + 5^2}{2(x^2-25)} \\[1em]
      &= \frac{(x-5)^2}{2(x+5)(x-5)} \\[1em]
      &= \frac{(x-5)}{2(x+5)} \\[1em]
    \end{align*}
  \end{sol}
\item $\dfrac{3x^2-3}{4x^2 -16}$
  \begin{sol}
    \begin{align*}
      \dfrac{3x^2-3}{4x^2 -16} =
      &= \frac{3(x^2-1)}{4(x^2-4)} \\[1em]
      &= \frac{3(x+1)(x-1)}{4(x+2)(x-2)} \\[1em]
    \end{align*}
    Inga gemensamma faktorer. Kan ej förenklas något mer.
  \end{sol}
\item $\dfrac{x^3y}{\dfrac{x}{y}} - x^2y^2$
  \begin{sol}
    \begin{align*}
      \dfrac{x^3y}{\dfrac{x}{y}} - x^2y^2 =
      &= \dfrac{x^3y}{\dfrac{x}{y}}\cdot \frac{y}{y} - x^2y^2 \\[1em]
      &= \dfrac{x^3y^2}{x} - x^2y^2 \\[1em]
      &= x^2y^2 - x^2y^2 \\[1em]
      &= 0.
    \end{align*}
  \end{sol}
\end{enumerate}
\paragraph{3.} Lös ekvationerna
\begin{enumerate}[(a)]
\item $\dfrac{4}{5x} - \dfrac{3}{4x} = \dfrac{1}{3}$
  \begin{sol}
    \begin{align*}
      \dfrac{4}{5x} - \dfrac{3}{4x} &= \dfrac{1}{3} \\[1em]
      \dfrac{4}{5x} - \dfrac{3}{4x} - \dfrac{1}{3} &= 0
                                                     \quad
                                                     \text{minsta
                                                     gemensamma
                                                     nämnare är
                                                     $3\cdot 4\cdot 5x$
                                                     }\\[1em]
      \dfrac{4}{5x} \cdot \frac{12}{12}  - \dfrac{3}{4x} \cdot
      \frac{15}{15} - \dfrac{1}{3}\cdot \frac{20x}{20x} &= 0 \\[1em]
      \frac{48-45-20x}{60x} =  0 \\[1em]
      \frac{3-20x}{60x} =  0
    \end{align*}
    För att vänsterledet ska bli noll räcker det med att täljaren blir
    noll.
    \begin{align*}
      3-20x &= 0 \\[1em]
      20x = 3 \\[1em]
      x = \frac{3}{20} \\[1em]
    \end{align*}
    Lösningen till ekvationen är $x = \dfrac{3}{20}$.
  \end{sol}
\item $\dfrac{x-2}{4} - \dfrac{2x+3}{3} + \dfrac{x+15}{6} = 0$
  \begin{sol}
    Minsta gemensamma nämnare är $12$. Det ger att
    \begin{align*}
      \dfrac{x-2}{4} - \dfrac{2x+3}{3} + \dfrac{x+15}{6} &= 0 \\[1em]
      \dfrac{x-2}{4}\cdot \frac{3}{3}-{2x+3}{3}\cdot \frac{4}{4} +
      \dfrac{x+15}{6}\cdot \frac{2}{2} &= 0 \\[1em]
      \frac{(3x-6) - (8x+12) +(2x+30)}{12} &= 0 \\[1em]
      \frac{-3x+12}{12} &= 0 \\[1em]
      -3x+12 &= 0 \\[1em]
      3x &= 12 \\[1em]
      x &= \frac{12}{3} = 4 \\[1em]
    \end{align*}
    Lösningen till ekvationen är $x = 4$.
  \end{sol}
\item $\dfrac{5}{x} - 4 = x$
  \begin{sol}
    \begin{align*}
      \dfrac{5}{x} - 4 &= x \\[1em]
      \dfrac{5}{x} - 4 -x &= 0 \\[1em]
      \dfrac{5}{x} - 4\cdot \frac{x}{x} -x\cdot \frac{x}{x} &= 0 \\[1em]
      \dfrac{5-4x-x^2}{x} &= 0 \\[1em]
      \dfrac{x^2 + 4x -5}{x} &= 0 \\[1em]
    \end{align*}
    Vi undersöker när täljaren blir lika med noll. Med $pq$-formeln
    får vi att
    \begin{align*}
      x &= -2 \pm \sqrt{2^2 + 5} \\[1em]
      x &= -2 \pm \sqrt{9} \\[1em]
      x &= -2 \pm 3.
    \end{align*}
    Lösningarna till ekvationen är $x = -5$ och $x = 1$
  \end{sol}
\item $\dfrac{1}{x+3} - \dfrac{2}{x} = 2$
  \begin{sol}
    \begin{align*}
      \dfrac{1}{x+3} - \dfrac{2}{x} &= 2 \\[1em]
      \dfrac{1}{x+3} - \dfrac{2}{x} -2 &= 0 \\[1em]
      \dfrac{1}{x+3}\cdot \frac{x}{x} - \dfrac{2}{x}\cdot
      \frac{x+3}{x+3} -2\cdot \frac{x(x+3)}{x(x+3)} &= 0 \\[1em]
      \frac{x- 2(x+3) - 2(x+3)x}{x(x+3)}&=0 \\[1em]
      \frac{-2x^2-7x-6}{x(x+3)}&=0 \\[1em]
    \end{align*}
    Vi undersöker återigen när täljaren är lika med noll, dvs vi löser
    ekvationen
    \[
      x^2 +\frac{7}{2} +3 = 0
    \]
    $pq$-formeln ger
    \begin{align*}
      x &= - \frac{7}{4} \pm \sqrt{\left(\dfrac{7}{4} \right)^2 -3} \\[1em]
      x &= - \frac{7}{4} \pm \sqrt{\dfrac{49}{16} -\dfrac{48}{16}} \\[1em]
      x &= - \frac{7}{4} \pm \sqrt{\dfrac{1}{16}} \\[1em]
      x &= - \frac{7}{4} \pm \frac{1}{4} \\[1em]
    \end{align*}
    Lösningarna till ekvationen är $x = -2$ och $x = -\dfrac{3}{2}$
  \end{sol}
\end{enumerate}
\paragraph{4.} Bestäm minsta gemensamma nämnare och förenkla uttrycket
\[
\frac{x}{2} +\frac{x}{4}+\frac{x}{6}+\frac{x}{25}+\frac{x}{30}
\]
\begin{sol}
  Vi har att
  \begin{align*}
    2 &= 2 \\
    4 &= 2\cdot 2\\
    6 &= 2\cdot 3\\
    25&= 5\cdot 5\\
    30&= 5\cdot 6 = 2\cdot 3 \cdot 5.
  \end{align*}
  Minsta gemensamma nämnare blir $2\cdot 2\cdot 3 \cdot 5 \cdot 5 =
  300$. Det ger att 
\begin{align*}
&\frac{x}{2} +\frac{x}{4}+\frac{x}{6}+\frac{x}{25}+\frac{x}{30} \\[1em]
&= \frac{150x}{300}
  +\frac{75x}{300}+\frac{50x}{300}+\frac{12x}{300}+\frac{10x}{300} \\[1em]
  &= \frac{297x}{300}.
\end{align*}
\end{sol}

\end{document}

%%% Local Variables:
%%% mode: latex
%%% TeX-master: t
%%% End:
